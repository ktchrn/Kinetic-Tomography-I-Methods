\subsection{Checking the KT solution}
\label{sec:KT-validation}

The \Reid{} HMSFRs (see Section DATA.3) are embedded in and, presumably, moving with dense molecular gas and can therefore serve as point-probes of the ISM's velocity field. 
{\bf REVISE NEXT SENTENCE} By comparing the HMSFRs' observed line-of-sight velocities to those a given KT technique assigns to their locations in PPD space, we can determine the accuracy of that technique.
By comparing the relative accuracies of the regularized and unregularized KT solutions, we can get a sense of the effectiveness of our regularization scheme.

{\bf REVISE THIS PARAGRAPH; SMOOTHER TRANSITION; DISTRIBUTION MENTIONED EARLIER; MAYBE AN EQUATIONS}
While the $\glon$ and $\glat$ of an HMSFR from \ are known to effectively arbitrary precision, its distance is only known to within about 10\%. 
To take this distance uncertainty into account when assigning a line-of-sight velocity to an HMSFR based on a KT solution, we take an average, weighted by the HMSFRs distance probability density function, of the KT solution's $\vlos(d)$ profile towards the HMSFR's $\glon$ and $\glat$. 
A similar procedure gives us an estimate of the uncertainty, expressed as a standard deviation, of this $\vlos$ value. 
When computing the line-of-sight velocity of an HMSFR according to a rotation curve, we ignore this distance uncertainty and simply use the best-fit distance. 

{\bf USE CLEMENS; PUT IN THAT IT'S MAYBE A MORE ACCURATE DESCRIPTION OF THE MOTION OF THE ISM IN THE FIRST QUADRANT AS USED BY, E.G. ROMAN-DUVAL (THE ONE ABOUT THE MOLECULAR FRACTION)}

In Figure {\bf MASER COMP}, we show a comparison of the HMSFRs' measured line-of-sight velocities and line-of-sight velocities assigned based on the CITET CLEMENS rotation curve, unregularized KT, and regularized KT.
There is a very clear improvement from a rotation-curve to either version of KT and a slight but noticeable improvement from unregularized to regularized KT. 
To get a crude quantitative estimate of the effect of regularization, we can compute the reduced $\chi$-squared values of the two sets of velocity estimates. 
If we assume the regularization parameter counts against the number of degrees of freedom, the reduced $\chi$-squared values of the unregularized and regularized KT solutions are 5 and 3, respectively. 

Much of the total discrepancy between the regularized KT solution and the HMSFR measurements is driven by 5 catastrpphic outliers. 
If we remove these 5 outliers, leaving 94 HMSFRs, the reduced $\chi$-squared values of the unregularized and regularized KT solutions drop to 4 and 1.3, respectively. 
We consider the advantage of regularized over unregularized KT to be sufficient to adopt the regularized KT solution as \emph{the} KT solution, and will refer to it as such below.

At the positions of 94 of 99 HMSFRs, at least, KT performs remarkably well. 
Next, we discuss the 5 catastrophic outliers.
