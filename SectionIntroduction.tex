\section{Introduction}
Many open problems in star formation, molecular cloud evolution, and galaxy-scale gas dynamics remain open because it has not been possible to measure the most useful quantities for resolving them -- the 3D gas velocity vector and 3D gas density over an extended area of sky. A measurement of these fields would allow us to solve the continuity equation \citep{euler1757principes}, and derive the rate at which density is changing across the Galaxy over a range of physical scales. 

The formation of giant molecular clouds (GMCs), for instance, is in part a matter of collecting a large mass in a small volume. 
By looking for sites at which gas flows are converging, it may be possible to find currently forming GMCs.
Conversely, one could look for diverging flows to detect and characterize the feedback-driven disruption of GMCs.
With sufficient spatial and velocity resolution, it would be possible to distinguish between the many theories of exactly how the necessary mass is accumulated and converted to cold, molecular gas. 
These theories make different predictions for the properties of the required converging flows.
For example, some theories assume the converging flows consist mostly of neutral hydrogen (\atomH) and invoke different sorts of instabilities at the collision interface of the flows to explain how this gas is rapidly converted to cold molecular hydrogen (\molH) \citep[e.g. ][]{Heitsch06,Clark:2012bq,2014ApJ...790...37C}.
Other theories assume that GMCs form from the collision and agglomeration of smaller molecular cloudlets \citep[e.g. ][]{Roberts:1987eb,Dobbs:2008ez,Tasker:2009gc}. 
One could directly distinguish between these two possible modes of GMC formation by determining whether gas that is converging on forming GMCs is predominantly neutral or molecular.
Without the ability to measure the velocity field as a function of all three spatial dimensions, it is difficult to even determine where converging flows are present.

The origin of the converging flows invoked above is also a matter of active interest.
In some theories, the self-gravity of a modest overdensity can be sufficient to induce collapse \citep{Kim:2002da,VazquezSemadeni:2007cj,2012MNRAS.425.2157D}.
Others invoke spatially coherent flows driven by feedback from star formation \citep{Fujimoto:2014kh} or perturbations in the Galactic potential such as spiral arms \citep{Roberts:1972bp,Bonnell:2006hn}.
Strong, shocked flows driven by spiral arms have been seen in strongly tidally interacting two-arm spiral galaxies in the nearby universe \citep{Visser:1980ud,Visser:1980vc,Shetty_2007} using $\CO$ and $\atomH$ observations.
Unfortunately, the resolution in $\atomH$ required to map these spiral shocks beyond very nearby galaxies with extreme two-armed spiral structure, is not observationally feasible \citep{Visser:1980ud}.

The nearest spiral galaxy is, of course, our own Milky Way. 
Studying the kinematics of the Milky Way replaces the problem of insufficient sensitivity and spatial resolution with the problem of confusion --- from our vantage point, it is difficult to determine how the ISM is moving as a function of 3D position.
In particular, it is essentially impossible to obtain the transverse velocity field of the ISM. Transverse velocities can only be derived from proper motions, which are difficult to measure for the diffuse and continuous ISM.
For many of the open problems we have outlined, even a measurement of the line-of-sight (radial) velocity as a function of 3D position (the 1-velocity field) would represent a significant step forward. 
While only having one component of the velocity field does make it difficult to empirically measure the  rate of inflow and outflow of matter relative to any given structure, it should still be possible to empirically measure this rate statistically across a sample of structures. 

No single observable can simultaneously measure the density and line-of-sight velocity at each location in the ISM, a construct we call the position-position-distance-velocity (PPDV) 4-cube. 
Instead, we have a number of observables of the ISM, each of which can be considered a (noisy and biased) projection of this PPDV 4-cube.
We dub the process of reconstructing this PPDV 4-cube from some set of observations ``Kinetic Tomography'' (KT). There are a number of existing kinds of data we can use. The first are classical radio observations of the ISM in both diffuse tracers (e.g. $\atomH$) and denser tracers (e.g. $\CO$). These are position-position-velocity (PPV) 3-cubes, a specific projection of the PPDV 4-cube, and have been classically used to infer the density of the ISM in 3-space assuming a Galactic rotation curve \citep[e.g.][ and references therein]{Levine_2006}. Another observable is the position-position-distance (PPD) reddening 3-cubes generated by examining the photometry of large numbers of stars and performing inference on the intervening dusty ISM. There has been dramatic progress in this field \citep{Marshall_2006,Lallement_2014,Green_2015}, which has been crucial for allowing this investigation. 
Interstellar absorption lines toward stars also represent a projection of the PPDV 4-cube and can simultaneously contain distance, column density, and velocity information about the intervening matter \citep{Welsh10,Zasowski_2014,2015MmSAI..86..521Z}.

There have been some attempts to construct maps or point estimates of $\vlos$ as a function of distance. 
Most such attempts have focused on individual spiral arms and used models of ISM flows around the spiral arms to directly invert PPV 3-cubes (e.g  \citealt{1972A&A....16..118S,Foster_2006}).
\citet{Reid_2016} have developed a different approach, which combines probability distributions from the standard kinematic distance, various geometric hints, and possible associations of emitting gas with structures that have parallax-based distance measurements into a combined syncretic probability distribution for the distance. 
Neither of these approaches uses the information available in reddening-based PPD 3-cubes. 

In the method we describe in this work, we use large-area CO and HI PPV 3-cubes and the \citet{Green_2015} (henceforth GSF) PPD 3-cube to reconstruct the ISM PPDV 4-cube. We perform a restricted version of the full tomographic reconstruction in which we assume each parcel of gas in the PPD 3-cube is assigned a single central line-of-sight velocity with some line-of-sight velocity width. This can be considered an inversion of the usual kinematic distance method, in which a line-of-sight velocity is converted to a distance using a Galactic rotation curve. Here we map distance along a sightline to velocity and allow deviations from a rotation curve in order to better match the distribution of matter in the PPV 3-cube.

In this work, we describe our method, the map it produces, and our evaluation of this map's accuracy and precision.
In Section \ref{sec:data}, we describe the datasets we use to make and evaluate our PPDV map.
In Section \ref{sec:KT}, we give a detailed explanation of the PPDV mapping technique and quantitatively demonstrate the accuracy of the technique's results.
In Section \ref{sec:discussion}, we discuss the broader accuracy and applicability of the technique.
We conclude in Section \ref{sec:conclusion}.
