\section{Introduction}
Many open problems in star formation, molecular cloud evolution, and galaxy-scale gas dynamics remain open because it has not been possible to measure the most useful quantities for resolving them -- the 3D gas velocity vector and 3D gas density over an extended area of sky. A measurement of these fields would allow us to solve the continuity equation \cite{euler1757principes}, and measure the rate at which density is changing in any region of the Galaxy, and over what physical scale. 

Flows of converging gas are a central part of theories of the formation of giant molecular clouds (GMCs) \cite{Vazquez_Semadeni_2007,Audit_2005}. Suggested mechanisms for forming GMCs out of diffuse gas include different sorts of thermodynamic, hydrodynamic, and magnetohydrodynamic instabilities, particularly in colliding flows \citep{Clark:2012bq,2014ApJ...790...37C,Heitsch06}; gas compression driven by large-scale structures such as spiral arms and expanding supergiant shells \citep{Roberts:1972bp,Bonnell:2006hn,Fujimoto:2014kh}; and collapse due to self-gravity \citep{Kim:2002da,2012MNRAS.425.2157D,VazquezSemadeni:2007cj}.
There is another set of mechanisms in which small, dense cloudlets form by one of the possibilities listed above and then grow, by cloudlet-cloudlet agglomeration \citep{Roberts:1987eb,Dobbs:2008ez,Tasker:2009gc} or diffuse interstellar medium (ISM) accretion \citep{Goldbaum:2011kj,Heitsch:2013jp}, into GMCs. It is very hard to distinguish observationally between these mechanisms using the standard observable of molecular clouds, emission maps from molecular tracers like carbon monoxide (CO) because the position of distance to gas along the line of sight can not be measured independently of its velocity. A method that could independently measure both the distance and the velocity of these flows could make significant progress in discriminating between these many theories of converging flows.

The importance of gas flows is not restricted to the areas immediately surrounding GMCs. Gas flows on much larger scales, driven by spiral arms, may dramatically effect the pattern of galaxy-wide star formation \cite{Roberts_1972,Bonnell_2006}. Spiral shocks have been seen in strongly tidally interacting two-arm spiral galaxies in the nearby universe \cite{Visser:1980vc,Visser:1980ud,Shetty_2007} using CO and neutral hydrogen (HI) observations, and have been implicated as a critical aspect of the the star formation process. Unfortunately, the resolution in HI required to test these theories, beyond very nearby galaxies with extreme two-armed spiral structure, is not yet observationally feasible \cite{Visser:1980ud}. Thus on larger scales we also need new methods to answer critical questions about how galaxies form molecular clouds and stars. 

How can we construct a method to measure this critical quantity? Since transvere velocities have been historically derived from proper motions, which cannot be reliably measured for any phase of the ISM, true 3D velocity fields are likely to remain inaccessible. 
For many of these open problems we have outlined, a measurement of the line-of-sight (radial) velocity as a function of 3D position, the 1-velocity field, would represent a huge step forward. With these measurements the inflow and outflow from structures can be measured statistically, and in the context of established models, if not fully empirically. 

No single observable can simultaneously measure the distance, density, and radial velocity to each parcel of gas in the interstellar medium, a construct we call the position-position-distance-velocity (PPDV) 4-cube. Instead, we have a number of observables of the ISM, each of which can be considered a (possibly lossy) projection of this PPDV cube. We dub the process of reconstructing this PPDV 4-cube from some set of observations (projections) ``Kinetic Tomography''. There are a number of existing kinds of data we can use. The first are classical radio observations of the ISM in both diffuse tracers (HI) and denser tracers (CO). These are position-position-velocity (PPV) cubes, a specific projection of the PPDV 4-cube, and have been classically used to infer the density of the ISM in 3-space assuming some kind of Galactic rotation curve \citep[e.g.][]{Levine_2006}. Another observable is the position-position-distance (PPD) reddening cubes generated by examining the photometry of large numbers of stars and performing inference on the intervening dusty ISM. There has been dramatic progress in this field \cite{Marshall_2006,Lallement_2014,Green_2015}, which is key to the present investigation. Interstellar absorption lines toward stars also represent a projection of the PPDV 4-cube, and can simultaneously contain distance information, column density information, and velocity information about the intervening gas \cite{Welsh10,Zasowski_2014,2015MmSAI..86..521Z}.

There have been some attempts to construct if not PPDV 4-cubes then maps or point estimates of $\vlos$ as a function of distance. 
Most such attempts have focused on individual spiral arms and used models of ISM flows around the spiral arms to try to directly invert PPV data (e.g  \citealt{1972A&A....16..118S,Foster_2006}).
\citet{Reid_2016} (henceforth \Reid{}) have developed a very different approach, which combines probability distributions coming from the standard kinematic distance, various geometric hints, and possible associations of emitting gas with structures that have parallax-based distance measurements into a single sort of syncretic probability distribution for the distance. 
Neither of these approaches uses the information available in reddening-based PPD cubes. 
Conversely, the method we use does not incorporate the sort of information these techniques are based on.

In the method we describe in this work, we use large-area CO and HI PPV cubes and the \citet{Green_2015} PPD cube to reconstruct the ISM PPDV 4-cube. We perform a restricted version of the full tomographic reconstruction in which we assume each parcel of gas in the PPD cube is assigned to a single central radial velocity with some radial velocity width. This can be considered an inversion of the usual kinematic distance method, in which a radial velocity is converted to a distance using a Galactic rotation curve. Here we convert each distance along a line of sight to a velocity, but instead of forcing a given rotation curve, we allow deviations from a rotation curve to best fit the PPV data cube.

In this work, we describe our method, the map it produces, and our evaluation of this map's accuracy and precision.
In Section \ref{sec:data}, we describe the datasets we use to make and evaluate our PPPV map.
In Section \ref{sec:methods}, we give a detailed explanation of the PPPV mapping technique and quantitatively demonstrate the accuracy of the technique's results.
In Section \ref{sec:discussion}, we discuss the broader accuracy and applicability of the technique.
In Section \ref{sec:conclusion}, we conclude.
