\section{Kinetic tomography}
\label{sec:KT}
We have developed a procedure for deriving the distribution of interstellar matter in PPDV space from measurements of its distribution in PPD and PPV space. 
The technical term for deriving a multi-dimensional distribution from lower-dimensional measurements is \emph{tomographic reconstruction}; in this context, lower-dimensional measurements are called \emph{projections}. 
Tomographic reconstruction from two projections is, in general, not possible -- there are many more independent variables than observational constraints. 
Our procedure for solving this specific case of the tomographic reconstruction problem uses simplifying assumptions about the structure of the ISM in PPDV space to reduce the number of independent variables.

For motivation, we first examine the assumptions behind a common technique for reconstructing a PPDV-space ISM distribution from PPV-space measurements alone. 
This is the widely known kinematic distance method, which maps line-of-sight velocities to distances based on an assumed rotation curve and Galactic geometry. 
While the kinematic distance method is usually presented as a way of converting a PPV-space distribution to a PPD-space, rather than PPDV-space, distribution, the two conversions are equivalent according to the kinematic distance method's underlying assumptions.
These assumptions can be combined into a single statement: (1) a location in PPD space can be assigned a single line-of-sight velocity (2) according to an assumed rotation curve and Galactic geometry. 
With these assumptions, it follows that knowing the PPD-space distribution of the ISM is equivalent to knowing its PPDV-space distribution. 

We use PPD measurements in addition to PPV measurements, allowing us to relax these assumptions. 
Our version: (1) a parcel of interstellar matter in PPD space can be assigned a \emph{Gaussian distribution} of line-of-sight velocities (2) whose center is \emph{within a fixed range of} the line-of-sight velocity predicted by an assumed rotation curve and Galactic geometry. 
Here, a ``parcel'' of interstellar matter refers to the contents of a single PPD 3-cube voxel. 
That is to say, we aim to assign a central line-of-sight velocity $\vlos$ and a line-of-sight Gaussian velocity width $\sigma_v$ to each voxel in the PPD 3-cube. 
From our assumptions, a description of the ISM in PPDV space consists of a description of its PPD-space distribution (the observed PPD 3-cube), a line-of-sight central velocity 3-cube in PPD space ($\vlos(\glon, \glat, d)$), and a line-of-sight velocity width 3-cube in PPD space ($\sigma_v (\glon, \glat, d)$). 
Thus, we have reduced our original problem of finding a PPDV 4-cube which is consistent with our observed PPD and PPV cubes to finding a $\vlos(\glon, \glat, d)$ and $\sigma_v(\glon, \glat, d)$ pair consistent with our PPV observations.


\subsection{Formalism}
\label{sec:KT-method}
We are looking for a $\vlos(\glon, \glat, d)$ and $\sigma_v(\glon, \glat, d)$ pair that, when combined with the observed PPD 3-cube, best matches the observed PPV 3-cube. 
This is an optimization problem. 
An optimization problem needs an objective function, and an objective function needs a quantity to compare to the data. 
In our case, that quantity is a model PPV 3-cube.
To obtain a model PPV 3-cube, we produce a PPDV 4-cube and then integrate it along the distance axis. 

Suppose that we have a matter density $\rho(\glon, \glat, d)$ in PPD space and a $\vlos(\glon, \glat, d)$ and $\sigma_v(\glon, \glat, d)$ pair. 
The matter density at a point $(\glon, \glat, d, v)$ in PPDV space is 
\begin{equation}
 \label{eqn:ppdv_cont}
  \rho(\glon, \glat, d, v) = \frac{\rho(\glon, \glat, d)}{\sqrt{2 \pi \sigma_v(\glon, \glat, d)^2}} 
   \, \exp\left(- \frac{\left( v - \vlos(\glon, \glat, d) \right)^2}{2 \sigma_v(\glon, \glat, d)^2} \right),
\end{equation}
and the matter density at a point $(\glon, \glat, v)$ in PPV space is
\begin{equation}
  \label{eqn:ppv_cont}
  \rho(\glon, \glat, v) = \int_0^{d_{max}} \rho(\glon, \glat, d, v) \, {\rm d}d.
\end{equation}

These equations apply for continuous matter density and $\vlos$ fields.
In our case, all quantities are defined on discrete grids. 
Instead of a PPD-space matter density $\rho(\glon, \glat, d)$, for example, we have a PPD 3-cube whose entries are total masses of protons $m(x, y, i)$ inside a voxel centered on $\glon_x$, $\glat_y$, and $d_i$.

To move from the continuous case to the discrete case, we assume that $\vlos(\glon, \glat, d)$ and $\sigma_v(\glon, \glat, d)$ are constant across a voxel and work with integrated quantities instead of densities. 
Equation \ref{eqn:ppdv_cont} becomes
\begin{equation}
  \label{eqn:ppdv_disc}
  m(x, y, i, j) = m(x, y, i) \, \int_{v_j - \Delta v}^{v_j + \Delta v} 
 \frac{1}{\sqrt{2 \pi \sigma_v(x, y, j)^2}} \,
    \exp\left(- \frac{\left( v - \vlos(x, y, j) \right)^2}{2 \sigma_v(x, y, j)^2} \right)\, {\rm d}v,
\end{equation}
where $2 \Delta v$ is the length of a voxel along the velocity axis.
Equation \ref{eqn:ppv_cont} becomes
\begin{equation}
  \label{eqn:ppv_disc}
  m(x, y, j) = \sum_{i} m(x, y, i, j).
\end{equation}
A cartoon representation of this procedure is shown in the top panel of Figure \ref{fig:diagram}.
In the cartoon, $\vlos(x, y, i)$ is assumed to be set by a rotation curve; the resulting $m(x, y, j)$ does not match the cartoon PPV data.
This mismatch is also generally true of the actual data.
Assuming a rotation curve and propagating it through produces a model PPV 3-cube which is clearly inconsistent with the observed PPV 3-cube. 

We quantify the discrepancy between a model PPV 3-cube, $m(x, y, j)$, and the observed PPV 3-cube, $PPV(x, y, j)$, with the objective function
\begin{equation}
  \label{eqn:obj_unreg}
  \mathcal{L}_u \left(\vlos(x, y, i), \sigma_v(x, y, i) \right) = 
 \frac{1}{2} \sum_{x, y, j} \left(PPV(x, y, j) -  m(x, y, j)\right)^2,
\end{equation}
the sum of square differences between the model and the observations. 
This is the \emph{unregularized} objective function; hence the subscript $u$ in $\mathcal{L}_u$.

We optimize $\mathcal{L}_u$ by varying the entries of $\vlos(x, y, i)$ and $\sigma_v(x, y, i)$. 
We restrict $\sigma_v(x, y, i)$ to be between $1$ and $15\text{ km sec}^{-1}$. 
The lower bound is set to be slightly smaller than $2 \Delta v$, which in this case is $1.4 \text{ km sec}^{-1}$, as there is very little difference between pixel-convolved Gaussians with standard deviations smaller than $\Delta v$. 
We restrict $\vlos(x, y, i)$ to be within $45 \text{ km sec}^{-1}$ of the line-of-sight velocity at $\glon_x$, $\glat_y$, and $d_i$ corresponding to the IAU-recommended $220 \text{ km sec}^{-1}$ flat rotation curve and an $8.5 \text{ kpc}$ separation between the Sun and the Galactic center.
The middle panel of Figure \ref{fig:diagram} shows a cartoon solution to the the $\mathcal{L}_u$ optimization problem. 
Qualitative features of the cartoon such as the close match between the model and observed PPV 3-cubes, the magnitudes of deviations from the rotation curve, and the spatial coherence of deviations from the rotation curve are also typical of the actual solution to the unregularized problem.

This solution is not necessarily unique. 
While we have reduced the number of parameters in the problem by making assumptions about the structure of the ISM in PPDV space, there are still situations in which different $\vlos(x, y, i)$ 3-cubes produce equivalent $m(x, y, j)$ 3-cubes. 
For example, if $m(x_0, y_0, i_0) = m(x_0, y_0, i_1)$, then $\vlos(x_0, y_0, i_0)$ and $\vlos(x_0, y_0, i_1)$ are interchangeable.

To deal with this problem, we introduce external information.
The $\glon$ and $\glat$ extent of the voxels in the 3-cubes and 4-cubes are often smaller than coherent structures such as molecular clouds. 
We may therefore expect voxels that share an $\glon$ or $\glat$ boundary to have similar line-of-sight velocities.
We encode this expectation into a regularized objective function, $\mathcal{L}_r$, by adding a term penalizing large differences between the line-of-sight velocities of voxels with shared $\glon$ or $\glat$ boundaries:
\begin{equation}
  \label{eqn:obj_reg}
  \begin{split}
  \mathcal{L}_r\left(\vlos(x, y, i), \sigma_v(x, y, i) \right) = 
  \mathcal{L}_u \left(\vlos(x, y, i), \sigma_v(x, y, i) \right) \\
  &+ \frac{\lambda}{2} \sum_{x, y} \left( \vlos(x + 1, y, i) - \vlos(x, y, i) \right)^2 \\
  &+ \frac{\lambda}{2} \sum_{x, y} \left( \vlos(x, y + 1, i) - \vlos(x, y, i) \right)^2,
  \end{split}
\end{equation}
where $\lambda$ is a parameter that sets the strength of the regularizing term. 
The bottom panel of Figure \ref{fig:diagram} shows a cartoon representation of a regularized solution. 
The regularization is represented by springs. 
As before, the qualitative features of the cartoon are realistic. 

We evaluate the accuracy of the unregularized and regularized KT-derived $\vlos$ cubes in the next section. 
For clarity, we will refer to the discretized cubes as $\vlos(\glon, \glat, d)$. 
