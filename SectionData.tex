\section{Data}
\label{sec:data}
\subsection{HI and CO data}

Radio emission lines of $\atomH$ and $\CO$ trace the two dominant constituents of the Galactic ISM, atomic and molecular gas. 
Ionized phases of the ISM do not contribute significantly to the column density and will therefore contribute negligibly to the extinction measured in \citet{Green_2015}. 
21-cm line emission from the hyperfine transition of $\atomH$ is usually optically thin and its integral is an excellent tracer of $\atomH$ column:
\begin{equation}\label{XHI}
N_{HI} = 1.8 \times 10^{18} \rm cm^{-2} \frac{ \int T_B dv}{\rm K~km~s^{-1}}.
\end{equation}
When the 21-cm line becomes optically thick, equation \ref{XHI} will underestimate the $\atomH$ column. 
However, this mostly happens in $\molH$-dominated regions \citep{Goldsmith_2007}.

We trace molecular gas here using the 115 GHz 1-0 rotational transition of $\CO$. 
The integral of this emission line can be converted to a $\molH$ column density using the conversion factor \citep{Bolatto_2013}

\begin{equation}\label{XCO}
X_{CO} = 2.0 \times 10^{20} \rm cm^{-2} \frac{ \int T_B dv}{\rm K~km/s}.
\end{equation}

This conversion factor has a number of known weaknesses stemming from complex excitation and opacity effects and real variation in the relative population of $\CO$ and $\molH$ molecules. 
We will address the impacts of these weaknesses in \S \ref{sec:discussion-systematics}. 

For our $\CO$ data, we use the interpolated whole-Galaxy PPV cube provided by \citet{Dame_2001}. 
We post-process these data with a plus-shaped median smoothing kernel to eliminate single-pixel artifacts in the data. 
This filtering procedure changes the total amount of CO emission by about 5\% over the entire Galaxy. 
The cube covers the full range in $\glon$ and $-30^\circ$ to $+30^\circ$ in $\glat$. 
The cube has a radial velocity resolution of $1.3 {\rm \, km~s}^{-1}$ and a velocity range of $-320{\rm \,km~s} ^{-1} < V_{LSR} < 320 \rm{~km~s}^{-1}$. 
We find that the native resolution and PPV extent of these data are appropriate for our investigation and retain the exact pixelization of these data for our analysis. 

For our $\atomH$ data, we use a combination of three large-area Galactic $\atomH$ surveys. 
South of declination 0$^\circ$, we use data from the 16$^\prime$ resolution GASS survey \citep{Kalberla_2010}; from declination 0$^\circ$ to 38$^\circ$ we use unpublished data from the 4$^\prime$ resolution GALFA-HI survey \citep{Peek_2011} Data Release 2; North of 38$^\circ$ we default to the 36$^\prime$ resolution LAB Survey \citep{Kalberla_2005}. 
We regrid these data onto the $7.5^\prime \times 7.5^\prime \times 1.3 {\rm \, km~s}^{-1}$ pixels of the \citet{Dame_2001} $\CO$ map.

The $\atomH$ and $\CO$ emission cubes are converted to ${\rm N}(\atomH)$ and ${\rm N}(\molH)$ cubes using equations \ref{XHI} and \ref{XCO}; these two column density cubes are then added to make a single $\mathrm{N_H}$ data cube.


\subsection{Dust data}

Our extinction cube is derived from the GSF reddening data. 
GSF use PanSTARRS photometry of 800 million stars to infer the cumulative reddening along the line of sight in 6.8$^\prime$ (NSIDE=512 HEALPix) pixels.
The distance information provided by the \citet{Green_2015} map is in steps of half a distance modulus, from 63 pc to 63 kpc. 
We regrid these data onto the \citet{Dame_2001} $\glon$-$\glat$ grid and difference them in distance to find the reddening between each distance modulus. 
This differential reddening is then converted to an ${\rm N_H}$ using the factor measured in \citet{Peek_2013}, 

\begin{equation}
\mathrm{N_H} = E\left(B-V\right) 7 \times 10^{21} \frac{\rm cm^{-2} }{\rm mag}. 
\end{equation}


\subsection{High-mass star forming region data}
\label{sec:data-HMSFR}
To check the accuracy of our method, we need measurements of the line-of-sight velocities and distances of clouds of gas. 
\citet{Reid:2014km}(henceforth \Reid{}) have measured the line-of-sight velocities, proper motions, and trigonometric parallaxes of water and methanol masers associated with 103 high-mass star forming regions (HMSFRs). 
Of these 103, 99 fall inside the footprint of the GSF reddening data. 
We adopt the line-of-sight velocities, velocity uncertainties, and parallaxes of these 99 HMSFRs as stated in Table 1 of \Reid{}. 