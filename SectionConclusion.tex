\section{Conclusion}
\label{sec:conclusion}

In this work we developed a method for measuring the radial velocity of parcels of the interstellar medium of a measured distance, which we dubbed  Kinetic Tomography. We argued that this method is important as a tool for measuring converging and diverging flows around our Galaxy as well as for detecting large scale deviations from assumed rotation curves. The method takes as inputs the three-dimensional distribution of dust in our Galaxy measured from stellar photometry and the emission spectra from Galatic $\CO$ and $\atomH$. We developed a technique that assigns each 3D parcel of ISM from the dust map a line-of-sight velocity and line-of-sight velocity width, in order to best reproduce the observed $\CO$ and $\atomH$ data. We found that we can improve the fidelity of our solution by implementing Tikhonov regularization, effectively coupling the line-of-sight velocity of adjacent pixels. 

As a test of our method we compare our results to independent measurements of HMSFRs from \Reid{}, which contain both distances and line-of-sight velocity information. We find that of the 99 HMSFRs in the area of sky we study, 94 are consistent with our results and 5 are outliers, all of which lie near Galactic center. This consistency indicates our map is an accurate representation of the velocity field of the ISM, at least in denser regions consistent with HMSFRs. We also find qualitative consistency with the peculiar velocity maps of \citet{1993A&A...275...67B}. 

Here we conclude that KT can be a very powerful tool for the study of the velocity structure of the Galactic ISM. In future work, we will investigate what KT can tell us about the Galactic rotation curve, streaming motions within the Galactic disk, and the vertical structure of Galactic flows.


