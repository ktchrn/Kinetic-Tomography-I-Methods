\section{Conclusion}
\label{sec:conclusion}

In this work we developed a method for measuring the radial velocity of parcels of the interstellar medium of a measured distance, which we dub  Kinetic Tomography. We argued that this method is important as it can be used as a tool for measuring converging and diverging flows around our Galaxy as well as large scale deviations from assumed rotation curves. The method takes as inputs the measured three-dimensional distribution of dust in our Galaxy and the emission spectra from CO and HI. We developed a technique that attempts to assign each 3D parcel of gas from a dust map a radial velocity and radial velocity width, to best reproduce the observed CO and HI data. We found that we can improve the fidelity of our solution by implementing Tikhonov regularization, effectively coupling adjacent pixels. 

As a test of our method we compare our results to measurements of masers from \Reid{}, which have both distances and radial velocities. We find that of the 99 masers in our sample, 94 are consistent with our results and 5 are outliers, largely near Galactic center. 


