\label{fig:diagram} Diagrams explaining the inverse kinematic distance methods, unregularized Kinetic Tomography, and regularized Kinetic Tomography. In each diagram, we show a map of residuals from a flat rotation curve (left quarter-circle), mock PPD and PPV data along a single sightline $\left(x_0, y_0\right)$ (narrow vertical and horizontal panels), and the mean ($\vlos\left(x_0, y_0, i, j\right)$) and standard deviation ($\sigma_v\left(x_0, y_0, i, j\right)$) of the velocity along that sightline. The longitude, latitude, distance, and velocity axes are index by $x, y, i,$ and $j$; note that the residual map shows only a single latitude. The results of applying a fixed inverse kinematic distance method, unregularized Kinematic Tomography, and regularized Kinetic Tomography are shown in black, purple, and pink. Regularization is indicated in the bottom schematic with spring symbols. Our regularization method minimizes the velocity difference between the voxels the spring symbols connect. 