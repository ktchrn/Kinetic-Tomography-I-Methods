\section{Discussion}
\label{sec:discussion}

\subsection{Unmet assumptions and their potential consequences}
\label{sec:discussion-systematics}

The procedure described in Section \ref{sec:KT} is based on two assumptions about the shape of the ISM in PPDV space and the relation between the two input datasets.
The first assumption is that the distribution of the ISM, in absolute units (e.g. number of protons), in PPD and PPV space is a linear scaling of the input reddening and gas line emission cubes.
The second assumption is that PPD and PPV cubes are projections of the same region of PPDV space, or equivalently that all the ISM contained in one is also contained in the other.
The third assumption is that the PPV cube can be reproduced by reprojecting the PPD cube -- ``lifting'' to PPDV space and projecting the result back down -- to PPV space.
This lift is assumed to have a simple description, a one-to-one function convolved with a Gaussian kernel in velocity space.
The fourth assumption is that voxels in PPD space that share a $\glon$ or $\glat$ boundary have similar central line-of-sight velocities.

These assumptions fail to hold, to different degrees, in the actual case we are applying the procedure to. 
The reddening-to-matter and $\atomH$- and CO-to-matter conversion is not a single linear relationship due to both variations in physical quantities such as the dust-to-gas ratio and non-linearities in the function relating the amount of matter to the amount of reddening or flux (e.g. $\atomH$ self-absortion). 
The PPV cube includes matter on the far side of the Galaxy while the PPD cube stops at or before the level of the Galactic center.
The PPV cube cannot be a reprojection of the PPD cube if the two are not tracing the same matter. 
Furthermore, there will be voxels in the PPD cube where the lift to PPDV space is not the simple one we have assumed.
Finally, since the velocity field of the ISM does not have to be smooth or even continuous, the first due to the influence of departures from axisymmetry such as the Galactic bar and the second due to the fact that the ISM can contain shocks.

Despite the many ways in which these assumptions can and almost certainly do break, we have been able to empirically confirm that the solution is quite accurate (see Section \ref{sec:KT-validation}). 
Evidently, the KT procedure is robust to some bending of its underlying assumptions.
Judging by the clustering of outlier HMSFRs in the inner Galaxy, there are some parts of the Galaxy where the assumptions are bent enough to start to cause problems.
In particular, the inner Galaxy is particularly likely to have mismatches between the PPD and PPV and complicated, rapidly spatially variable velocity fields.
Even there, KT does not completely and utterly fail -- the error rate is $\sim 30 \%$, not $\sim 100 \%$.
The inner Galaxy falls into a gray region in the parameter space of assumption bending where there is just enough of it to noticeably, but not absolutely, degrade the solution quality.

The 30\% mentioned above is the error rate specifically among PPD voxels that contain HMSFRs.
These are not necessarily typical voxels. 
The error rate of an arbitrary inner Galaxy voxel could potentially be quite different. 
While we may expect the error rate of a voxel that contains an HMSFR and others near it to be similar, this is merely a heuristic argument.
This argument is also not useful for parts of the space that are not near HMSFRs.
The HMSFRs are all in dense molecular gas within a few tens of parsecs of the Galactic plane. There is very little we can say about the error rate in diffuse gas, especially at high Galactic latitudes. 

To get a quantitative and more broadly applicable estimate of the error rate, we would need either a less density-biased set of independent $\vlos(\glon, \glat, d)$ measurements or a set of artificial injection tests. 
By an "artificial injection test", we mean a numerical experiment in which we artificially observe a model galaxy's ISM, reconstruct the $\vlos$ field from these artificial observations using KT, and compare the reconstructed and input model $\vlos$ fields. 
To the best of our knowledge, there are no currently available catalogs of these sorts of less density-biased measurements, ruling out option one.
The steps involved in artificial injection tests, particularly simulating galaxies at sufficiently high resolution and producing artificial observations in a way that includes the non-trivial systematics in the actual observations, are complicated enough to put option two beyond the scope of this work.

Both of these options are plausible directions for future work. 
The 1.527 $\mu$m diffuse interstellar band (DIB), for instance, has been mapped over much of the northern sky by the APOGEE survey \citep{2015ApJ...798...35Z}. 
Observations of this DIB towards APOGEE stars with known distances could potentially be used to build catalogs or even maps of $\vlos(\glon, \glat, d)$ using an independent dataset. 
Artificial injection tests are conceptually straightforward, though they do require a substantial investment of time and computational resources.
While neither option is easy, we would argue that some combination of the two will be necessary before the KT-derived $\vlos$ map can be trusted away from the HMSFRs of \Reid.
